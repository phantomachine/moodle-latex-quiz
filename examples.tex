% !TEX program=xelatex
% !TEX options = --shell-escape -synctex=1 -interaction=nonstopmode -file-line-error "%DOC%"

\documentclass{article}
\usepackage{moodle}
%\usepackage[draft]{moodle}
\usepackage{minted}
\usepackage{python}

\begin{document}

\begin{quiz}{LaTeX-Moodle Quiz}

    %% MULTI environment ===============================================
    \begin{multi}[single]{Old-skool MCQ}
        Time is finite and indexed by $t \in \{0, 1, ...,T\}$. %
        Let the optimal value of a policy maker beginning with %
        resources $k_{0}$ be given by:
        $V_{0}(k_{0}) = %
        \max_{\{c_{t}, k_{t+1}\}_{t=0}^{T}}%
        \sum_{t=0}^{T}\beta^{t} (c_{t})^{\alpha}$ %
        subject to the constraints: %
        $k_{t+1} = \min\{k_{t},1\} - c_{t}$, %
        $0 \leq c_{t} \leq \min\{k_{t},1\}$, %
        and, $k_{T+1} \geq 0$, %
        where $\alpha \in (0,1)$; and $k=K/L$ and $c$, respectively, %
        refer to per-worker capital stock and consumption. %
        The state space $X \ni k_{t}$ is bounded.
        \\
        Describe precisely what we mean by a strategy in this setting. 
		\item* A strategy is a date and state contingent plan %
        $\{g_{t}(k_{t})\}_{t=0}^{T}$ such that %
        $c_{t} = g_{t}(k_{t})$ at each date $t$ and state $k_{t}$.
		\item A strategy is an optimal date and state contingent plan %
        $\{g_{t}(k_{t})\}_{t=0}^{T}$ such that %
        $c_{t} = g_{t}(k_{t})$ at each date $t$ and state $k_{t}$.
		\item A strategy is the optimal date and state contingent plan %
        $\{g_{t}(k_{t})\}_{t=0}^{T}$ such that $c_{t} = g_{t}(k_{t})$ %
        at each date $t$ and state $k_{t}$.
		\item A strategy is a policy selection $c_{t} = g_{t}(k_{t})$ %
        at each date $t$ and state $k_{t}$.
    \end{multi}
    %% =================================================================

    %% NUMERICAL environment ===========================================
    \begin{numerical}[tolerance=0.01]{Numerical}
        The approximate the value of $\sqrt{2}$ is
        \item[tolerance={0.01}] 1.4142
        \item[fraction=20, feedback={Yikes! Double!}] 0.70711
        \item[fraction=0, feedback={Wrong!}] *
        \item[fraction=0, feedback={Wrong!}] 2.5
    \end{numerical}
    %% =================================================================

    
    %% CLOZE environment ===============================================
    \begin{cloze}{A Cloze type question}
        Thanks to calculus, invented by Isaac
        \begin{shortanswer}[usecase]
            \item Newton
        \end{shortanswer},
        we know that the first derivative of $x^2$ is
        \begin{multi}[horizontal]
            \item* This one $2x$
            \item This one $3x$
            \item This one $0$
        \end{multi}
        and that $\int_0^2 x^2 dx$ equals
        \begin{numerical}
            \item[tolerance={0.001}] 2.667
        \end{numerical}.
        Thanks, Isaac!
    \end{cloze}
     % ==================================================================

    %% CLOZE with MULTI environment ====================================
    \begin{cloze}{Another Cloze type with in-line MCQ}
        ``Hello, Goodbye'' was a song by:
        \begin{shortanswer}[usecase]
            \item The Beatles
        \end{shortanswer}.
        \\
        They were sometimes high on
        \begin{multi}[horizontal]
            \item marijuana
            \item* LSD
            \item speed
            \item Molly
        \end{multi}
        It was recorded in
        \begin{shortanswer}[usecase]
            \item EMI Studios
        \end{shortanswer}
        in the city of
        \begin{multi}[vertical]
            \item Birmingham
            \item* London
            \item Liverpool
            \item Edinburgh
        \end{multi}
    \end{cloze}
    % ==================================================================

    %% SHORT ANSWER environment ========================================
    %% Admissible responses: allow for variation in inputs
    \begin{shortanswer}[usecase]{Short answer}
        Newton's rival was Gottfried Wilhelm \blank
        \item[fraction=90] Leibniz
        \item[fraction=100] Leibniz.
        \item[fraction=70] leibniz
        \item[fraction=80] leibniz.
    \end{shortanswer}
    % ==================================================================

    %% ESSAY environment ===============================================
    %% Option: response format= html, file or html+file
    %% You may have to manually (re)set these later in ANU Wattle
    \begin{essay}[response format=html+file]{Essay, Examiner notes}
        Let $\beta\in(0,1)$ and $\phi := \phi_{t}$. 
        
        Prove that this monetary equilibrium $\phi = \beta\phi_{+1}$ 
        is unique.

        Show here:
        \item Examiner note 1
        \item Examiner note 2
        \item Examiner note 3  
    \end{essay}
    % ==================================================================

    %% PYTHON environment ==============================================
    %% Batch generate many numerical quizzes!
    \begin{python}
        for x in range(2,10):
          for y in range(2,10):
            if x > y:
              if x/y == x//y:
                points=4
              else:
                points=3  
              print(rf"""\begin{{cloze}}[points={points}]{{Arithmetic Quiz {(x,y)}}}
              Solve the following tasks!\\
              \begin{{numerical}}
              ${x} + {y} =$
              \item {x+y} 
              \end{{numerical}}\\
              \begin{{numerical}}
              ${x} - {y} =$
              \item {x-y} 
              \end{{numerical}}\\
              \begin{{numerical}}
              ${x} \cdot {y} =$
              \item {x*y} 
              \end{{numerical}}\\""")
              if x/y == x//y:
                print(rf"""\begin{{numerical}}
                ${x} : {y} =$
                \item {x//y} 
                \end{{numerical}}\\""")
              print(rf"\end{{cloze}}")
    \end{python}
    % ==================================================================

    %% MATCHING environment =====================================
    \begin{matching}[shuffle=false]{Incomplete markets, HA, Aiyagari}
        \textbf{Practical (Basic, [10])}

        Consider a mashup of the model of Mark Huggett and Ayse Imhoroglu from our lab session. We use the same notation as in the lectures notes. These are available to you to consult.

        Suppose now there is a government that wishes to finance
        unemployment benefit transfer $b$ to the unemployed agents. Assume
        that aggregate labor and capital incomes, respectively $w_{t}N_{t}$ 
        and $r_{t}K_{t}$, are taxed at a flat rate $\tau$ 
        to finance the total transfer to all the unemployed. So the government budget constraint would be 

        \[
          (1-N)b=\tau\left(r_{t}K_{t}+w_{t}N_{t}\right),
        \]

        where $N$ is the equilibrium ergodic measure of employed agents.

        On the agents' side, the agents' sequential budget constraint would now be

        \[
            a_{t+1} + c_{t} = (1-\tau)w(r)e_{t} + b(1-e_{t}) + \left[1+(1-\tau)r\right]a_{t},
        \]

        where $a$ is the agent's asset position, $c$ is consumption and $e\in\{0,1\}$ can take on an unemployment (0) and employment (1) state, according to a Markov chain.

        Recall in this model, the market return $r$ will equal the firm's (net) marginal product of capital.

        The numbered steps below refer to sequentially ordered tasks to be assigned to the computer. Choose the correct task and match to its
        correct step number.

        %\textbf{Warning:} The answer order below may be randomized by the quiz!
        \item Step 1: \answer Guess a market return and tax rate.
        \item Step 2: \answer Solve agents' dynamic program.
        \item Step 3: \answer Compute distribution of asset claims on capital.
        \item Step 4: \answer Get implied aggregate capital and labor supply.
        \item Step 5: \answer Check market clearing and government budget balance.
        \item Step 6: \answer Update market return and tax rate.
        \item \answer Compute ergodic unemployment rate.
        \item \answer Update government spending level.
        \item \answer Set tax rate to zero.
        \item \answer Iterate until tax rate maximizes government budget.
        \item \answer Give a dog a bone.
        \item \answer Remember to pick up milk.
        \item \answer Calculate optimal tax and transfer.
    \end{matching}

\end{quiz}

%%%%%%%%%%%%%%%%%% FIN %%%%%%%%%%%%%%%%%%%%%%%%%%%
\end{document}

