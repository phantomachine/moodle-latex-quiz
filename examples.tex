% !TEX program=xelatex
% !TEX options = --shell-escape -synctex=1 -interaction=nonstopmode -file-line-error "%DOC%"

\documentclass{article}
\usepackage{moodle}
%\usepackage[draft]{moodle}
\usepackage{minted}
\usepackage{python}

\begin{document}

\begin{quiz}{LaTeX-Moodle Quiz}

    %% MULTI environment ===============================================
    \begin{multi}[single]{Old-skool MCQ}
        Time is finite and indexed by $t \in \{0, 1, ...,T\}$. %
        Let the optimal value of a policy maker beginning with %
        resources $k_{0}$ be given by:
        $V_{0}(k_{0}) = %
        \max_{\{c_{t}, k_{t+1}\}_{t=0}^{T}}%
        \sum_{t=0}^{T}\beta^{t} (c_{t})^{\alpha}$ %
        subject to the constraints: %
        $k_{t+1} = \min\{k_{t},1\} - c_{t}$, %
        $0 \leq c_{t} \leq \min\{k_{t},1\}$, %
        and, $k_{T+1} \geq 0$, %
        where $\alpha \in (0,1)$; and $k=K/L$ and $c$, respectively, %
        refer to per-worker capital stock and consumption. %
        The state space $X \ni k_{t}$ is bounded.
        \\
        Describe precisely what we mean by a strategy in this setting. 
		\item* A strategy is a date and state contingent plan %
        $\{g_{t}(k_{t})\}_{t=0}^{T}$ such that %
        $c_{t} = g_{t}(k_{t})$ at each date $t$ and state $k_{t}$.
		\item A strategy is an optimal date and state contingent plan %
        $\{g_{t}(k_{t})\}_{t=0}^{T}$ such that %
        $c_{t} = g_{t}(k_{t})$ at each date $t$ and state $k_{t}$.
		\item A strategy is the optimal date and state contingent plan %
        $\{g_{t}(k_{t})\}_{t=0}^{T}$ such that $c_{t} = g_{t}(k_{t})$ %
        at each date $t$ and state $k_{t}$.
		\item A strategy is a policy selection $c_{t} = g_{t}(k_{t})$ %
        at each date $t$ and state $k_{t}$.
    \end{multi}
    %% =================================================================

    %% NUMERICAL environment ===========================================
    \begin{numerical}[tolerance=0.01]{Numerical}
        The approximate the value of $\sqrt{2}$ is
        \item[tolerance={0.01}] 1.4142
        \item[fraction=20, feedback={Yikes! Double!}] 0.70711
        \item[fraction=0, feedback={Wrong!}] *
        \item[fraction=0, feedback={Wrong!}] 2.5
    \end{numerical}
    %% =================================================================

    
    %% CLOZE environment ===============================================
    \begin{cloze}{A Cloze type question}
        Thanks to calculus, invented by Isaac
        \begin{shortanswer}[usecase]
            \item Newton
        \end{shortanswer},
        we know that the indefinite integral of $x^2$ is
        \begin{multi}[horizontal]
            \item This one $2x$
            \item* This one $3x$
            \item This one $0$
        \end{multi}
        and that $\int_0^2 x^2 dx$ equals
        \begin{numerical}
            \item[tolerance={0.001}] 2.667
        \end{numerical}.
        Thanks, Isaac!
    \end{cloze}
     % ==================================================================

    %% CLOZE with MULTI environment ====================================
    \begin{cloze}{Another Cloze type with in-line MCQ}
        ``Hello, Goodbye'' was a song by:
        \begin{shortanswer}[usecase]
            \item The Beatles
        \end{shortanswer}.
        \\
        They were high on
        \begin{multi}[horizontal]
            \item marijuana
            \item* LSD
            \item speed
            \item Molly
        \end{multi}
        It was recorded in
        \begin{shortanswer}[usecase]
            \item EMI Studio
        \end{shortanswer}
        in the city of
        \begin{multi}[vertical]
            \item Birmingham
            \item* London
            \item Liverpool
            \item Edinburgh
        \end{multi}
    \end{cloze}
    % ==================================================================

    %% SHORT ANSWER environment ========================================
    %% Admissible responses: allow for variation in inputs
    \begin{shortanswer}[usecase]{Short answer}
        Newton's rival was Gottfried Wilhelm \blank
        \item[fraction=90] Leibniz
        \item[fraction=100] Leibniz.
        \item[fraction=70] leibniz
        \item[fraction=80] leibniz.
    \end{shortanswer}
    % ==================================================================

    %% ESSAY environment ===============================================
    %% Option response format= html, file or html+file
    \begin{essay}[response format=html+file]{Essay, Examiner notes}
        Let $\beta\in(0,1)$ and $\phi := \phi_{t}$. 
        
        Prove that this monetary equilibrium $\phi = \beta\phi_{+1}$ 
        is unique.

        Show here:
        \item Use contraction mapping
        \item To show contraction, use Blackwell
        \item Show solution space is complete ms   
    \end{essay}
    % ==================================================================

    %% PYTHON environment ==============================================
    %% Batch generate many numerical quizzes!
    \begin{python}
        for x in range(2,10):
          for y in range(2,10):
            if x > y:
              if x/y == x//y:
                points=4
              else:
                points=3  
              print(rf"""\begin{{cloze}}[points={points}]{{Arithmetic Quiz {(x,y)}}}
              Solve the following tasks!\\
              \begin{{numerical}}
              ${x} + {y} =$
              \item {x+y} 
              \end{{numerical}}\\
              \begin{{numerical}}
              ${x} - {y} =$
              \item {x-y} 
              \end{{numerical}}\\
              \begin{{numerical}}
              ${x} \cdot {y} =$
              \item {x*y} 
              \end{{numerical}}\\""")
              if x/y == x//y:
                print(rf"""\begin{{numerical}}
                ${x} : {y} =$
                \item {x//y} 
                \end{{numerical}}\\""")
              print(rf"\end{{cloze}}")
    \end{python}
    % ==================================================================

\end{quiz}

%%%%%%%%%%%%%%%%%% FIN %%%%%%%%%%%%%%%%%%%%%%%%%%%
\end{document}

